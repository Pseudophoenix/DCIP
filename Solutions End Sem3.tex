\documentclass{article}
\usepackage{amsmath}
\usepackage{enumitem}

\begin{document}

\section*{Encoding Techniques for Binary Data \texttt{100110111}}

To encode the binary data \texttt{100110111}, we will explore four different encoding techniques: \textbf{Bipolar-AMI}, \textbf{Pseudoternary}, \textbf{Manchester}, and \textbf{Differential Manchester}.

\begin{enumerate}[label=\alph*.]
    \item \textbf{Bipolar-AMI (Alternate Mark Inversion)}
    
    In Bipolar-AMI:
    \begin{itemize}
        \item A '0' is represented by no signal (i.e., zero voltage).
        \item A '1' is represented by alternating positive and negative voltages (e.g., $+V$ and $-V$).
    \end{itemize}

    For the binary string \texttt{100110111}:
    \begin{align*}
        \text{1} &\rightarrow +V \\
        \text{0} &\rightarrow 0 \\
        \text{0} &\rightarrow 0 \\
        \text{1} &\rightarrow -V \\
        \text{1} &\rightarrow +V \\
        \text{0} &\rightarrow 0 \\
        \text{1} &\rightarrow -V \\
        \text{1} &\rightarrow +V \\
        \text{1} &\rightarrow -V
    \end{align*}

    The encoded signal is: \texttt{+V, 0, 0, -V, +V, 0, -V, +V, -V}

    \item \textbf{Pseudoternary}
    
    In Pseudoternary encoding:
    \begin{itemize}
        \item A '0' is represented by alternating between positive and negative voltages.
        \item A '1' is represented by no signal (i.e., zero voltage).
    \end{itemize}

    For \texttt{100110111}:
    \begin{align*}
        \text{1} &\rightarrow 0 \\
        \text{0} &\rightarrow +V \\
        \text{0} &\rightarrow -V \\
        \text{1} &\rightarrow 0 \\
        \text{1} &\rightarrow +V \\
        \text{0} &\rightarrow -V \\
        \text{1} &\rightarrow 0 \\
        \text{1} &\rightarrow +V \\
        \text{1} &\rightarrow -V
    \end{align*}

    The encoded signal is: \texttt{0, +V, -V, 0, +V, -V, 0, +V, -V}

    \item \textbf{Manchester Encoding}
    
    In Manchester encoding:
    \begin{itemize}
        \item A '0' is represented by a transition from high to low (first half high, second half low).
        \item A '1' is represented by a transition from low to high (first half low, second half high).
    \end{itemize}

    For \texttt{100110111}:
    \begin{align*}
        \text{1} &\rightarrow \text{Low to High} \\
        \text{0} &\rightarrow \text{High to Low} \\
        \text{0} &\rightarrow \text{High to Low} \\
        \text{1} &\rightarrow \text{Low to High} \\
        \text{1} &\rightarrow \text{Low to High} \\
        \text{0} &\rightarrow \text{High to Low} \\
        \text{1} &\rightarrow \text{Low to High} \\
        \text{1} &\rightarrow \text{Low to High} \\
        \text{1} &\rightarrow \text{Low to High}
    \end{align*}

    The encoded signal is: \texttt{10, 01, 01, 10, 10, 01, 10, 10, 10}

    \item \textbf{Differential Manchester Encoding}
    
    In Differential Manchester encoding:
    \begin{itemize}
        \item A '0' is represented by a transition at the beginning of the bit period.
        \item A '1' is represented by no transition at the beginning of the bit period.
    \end{itemize}

    For \texttt{100110111}:
    \begin{align*}
        \text{Starting with Low:} & \\
        \text{1} &\rightarrow \text{No transition, remains Low} \\
        \text{0} &\rightarrow \text{Transition at the start, goes High} \\
        \text{0} &\rightarrow \text{Transition at the start, goes Low} \\
        \text{1} &\rightarrow \text{No transition, remains Low} \\
        \text{1} &\rightarrow \text{No transition, remains Low} \\
        \text{0} &\rightarrow \text{Transition at the start, goes High} \\
        \text{1} &\rightarrow \text{No transition, remains High} \\
        \text{1} &\rightarrow \text{No transition, remains High} \\
        \text{1} &\rightarrow \text{No transition, remains High}
    \end{align*}

    The encoded signal is: \texttt{Low, High, Low, Low, Low, High, Low, Low, Low}
\end{enumerate}

\section*{Difference Between Datagram and Virtual Circuit Operation}

\begin{description}
    \item[Datagram Operation:] 
    \begin{itemize}
        \item \textbf{Connectionless:} Each packet (datagram) is treated independently, and there is no need to establish a connection before sending packets.
        \item \textbf{Flexibility and Resilience:} Packets can take alternate routes to adapt to network changes.
        \item \textbf{Packet Delivery:} Delivery is not guaranteed, and packets may arrive out of order, be duplicated, or lost.
        \item \textbf{Example Protocols:} Internet Protocol (IP).
    \end{itemize}

    \item[Virtual Circuit Operation:]
    \begin{itemize}
        \item \textbf{Connection-oriented:} Establishes a dedicated path before communication begins, ensuring a reliable session.
        \item \textbf{Predictable Delivery:} All packets follow the same path, arriving in order and with reliable delivery.
        \item \textbf{Resource Reservation:} Resources may be reserved for the duration of communication, enhancing bandwidth and latency performance.
        \item \textbf{Example Protocols:} Frame Relay and X.25.
    \end{itemize}
\end{description}

\end{document}
